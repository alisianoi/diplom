\documentclass[12pt]{article}
\usepackage[a4paper]{geometry}

\usepackage[utf8]{inputenc}
\usepackage[T2A]{fontenc}

\usepackage[russian]{babel}
\usepackage{mathtools}
\usepackage{graphicx}
\usepackage{minted}

% do not show colored borders around links
\usepackage[hidelinks]{hyperref}

% make indentation at the beginning of sections
\usepackage{indentfirst}

% backend=biber is the default but /wo it there is a warning
% use the biblatex-gost package which provides gost support
\usepackage[backend=biber,bibstyle=gost-standard]{biblatex}
\bibliography{references}

% set a couple mathematical environments
\newtheorem{definition}{Определение}
\newtheorem{example}{Пример}

%\usepackage[ruled,section]{algorithm}
%\usepackage[noend]{algorithmic}
%\usepackage[all]{xy}

% Параметры страницы
\textheight=24cm
\textwidth=16cm
%% \oddsidemargin=5mm
%% \evensidemargin=-5mm
%% \marginparwidth=36pt
%% \topmargin=-1cm
%% \flushbottom
%% \raggedbottom
%% \tolerance 3000
%% % подавить эффект "висячих стpок"
%% \clubpenalty=10000
%% \widowpenalty=10000
%% \renewcommand{\baselinestretch}{1.1}
\renewcommand{\baselinestretch}{1.5} %для печати с большим интервалом

\begin{document}

\begin{titlepage}
\begin{center}
    Московский государственный университет имени М. В. Ломоносова

    \bigskip
    \includegraphics[width=50mm]{msu.eps}

    \bigskip
    Факультет Вычислительной Математики и Кибернетики\\
    Кафедра Математических Методов Прогнозирования\\[10mm]

    \textsf{\large\bfseries
        ДИПЛОМНАЯ РАБОТА СТУДЕНТА 417 ГРУППЫ\\[10mm]
        <<Обработка множеств логических закономерностей с помощью
        дисперсионного критерия>>
    }\\[10mm]

    \begin{flushright}
        \parbox{0.5\textwidth}{
            Выполнил:\\
            студент 4 курса 417 группы\\
            \emph{Лисяной Александр Евгеньевич}\\[5mm]
            Научный руководитель:\\
            д.ф-м.н., профессор\\
            \emph{Рязанов Владимир Васильевич}
        }
    \end{flushright}

    \begin{tabular}{p{0.45\textwidth}p{0.45\textwidth}}
        Заведующий кафедрой\newline
        Математических Методов\newline
        Прогнозирования, академик РАН
        &
        ~\newline~\newline
        \hfill\hbox to 0.45\textwidth{\hrulefill~Ю. И. Журавлёв}
    \\[20mm]
        К защите допускаю\newline
        \hbox to 0.4\textwidth{<<\hbox to 12mm{\hrulefill}>> \hrulefill~2015 г.}
        &
        К защите рекомендую\newline
        \hbox to 0.45\textwidth{<<\hbox to 12mm{\hrulefill}>> \hrulefill~2015 г.}
    \end{tabular}

    \vspace{\fill}
    Москва, 2015
\end{center}
\end{titlepage}

\newpage
\renewcommand{\contentsname}{Содержание}
\tableofcontents

\newpage
\begin{abstract}
    Данный документ является образцом оформления дипломной работы для
    студентов кафедры Математических методов прогнозирования ВМК~МГУ.
    Приведённые ниже рекомендации взяты из~статьи <<Написание отчётов
    и статей (рекомендации)>> на~вики"~ресурсе
    \texttt{www.MachineLearning.ru}.  Студенты, готовящие дипломную
    работу к~защите, могут найти много полезной информации также
    в~статьях <<Научно-исследовательская работа (рекомендации)>>,
    <<Подготовка презентаций (рекомендации)>>, <<Защита выпускной
    квалификационной работы (рекомендации)>> на~том~же ресурсе.

    Аннотация обычно содержит краткое описание постановки задачи
    и~полученных результатов, одним абзацем на 10--15 строк.  Цель
    аннотации "--- обозначить в~общих чертах, о~чём работа, чтобы
    человек, совершенно не~знакомый с~данной работой, понял,
    интересна~ли ему эта тема, и~стоит~ли читать дальше.  Аннотация
    собирается в~последнюю очередь путем легкой модификации наиболее
    важных и~удачных фраз из введения и~заключения.
\end{abstract}

\newpage
\section{Введение}

Во введении рассказывается, где возникает данная задача, и~почему её
решение так важно.  Вводится на неформальном уровне минимум терминов,
необходимый для понимания постановки задачи.  Приводится краткий
анализ источников информации (литературный обзор): как эту задачу
решали до сих пор, в~чем недостаток этих решений, и~что нового
предлагает автор.  Формулируются цели исследования.  В~конце введения
даётся краткое содержание работы по разделам; при этом отмечается,
какие подходы, методы, алгоритмы предлагаются автором впервые.
При~упоминании ключевых разделов кратко формулируются основные
результаты и~наиболее важные выводы.

Цель введения: дать достаточно полное представление о~выполненном
исследовании и~полученных результатах, понятное широкому кругу
специалистов.  Большинство читателей прочтут именно введение и, быть
может, заключение.

Во~введении автор решает сложную оптимизационную проблему: как
сообщить только самое важное, потратив минимум времени читателя,
да~так, чтобы максимум читателей поняли, о~чём вообще идёт речь.

Введение лучше писать напоследок, так как в~ходе работы обычно
происходит переосмысление постановки задачи.  Если же введение писать,
когда работа еще не~готова, задача усложняется вдвойне.  В~конце
обычно приходит понимание, что всё получилось совсем не~так, как
планировалось в~начале, и~исходный вариант введения всё равно придётся
переписывать.  Кстати, к~таким «потерям» надо относиться спокойно~---
в~хорошей работе почти каждый абзац многократно переделывается
до~неузнаваемости.

Введение имеет много общего с~текстом доклада на~защите, поэтому имеет
смысл готовить их одновременно.

\subsection{Определения и~обозначения}

%% Формальная постановка задачи. Для известных понятий желательно
%% придерживаться стандартных обозначений.

%% Общепринятые термины вводятся словом «называется». Термины,
%% придуманные самим автором, вводятся словами «назовём» или «будем
%% называть».  Обычно этот раздел заканчивается формальной постановкой
%% задачи.  Именно с~этого раздела стоит начинать писать работу.

% постановка задачи классификации Воронцовым

Общая логика: задача классификации -> логические закономерности -> SLS
у Воронцова --- два способа улучшить -> улучшить с помощью
кластеризации.

Пусть имеется пространство объектов \(X\) и конечное множество имен
классов \(Y = \left\{1, \dots, M\right\}\). Пусть также имеется
\emph{обучающая выборка} \(X^{l} = (x_i, y_i)_{i = 1}^{l}\), в которой
для каждого объекта \(x_i\) известен его класс \(y_i \in Y\).  Для
восстановления целевой зависимости \(y^{*}(x_i) = y_i\) построим
алгоритм классификации \(a\colon X \rightarrow Y\), аппроксимирующий
\(y^{*}\) на всём пространстве объектов \(X\).

Для решения такой задачи классификации (восстановления зависимости
\(y^{*}\) по обучающей выборке \(X^l\)) существует множество
разнообразных алгоритмов. Некоторые из них основаны на поиске
логических закономерностей. Рассмотрим эту группу алгоритмов
подробнее, при этом будем придерживаться обозначений
\cite{voron10logicalgs}.

\begin{definition}
  Функция \(\varphi \colon X \rightarrow \left\{0, 1\right\}\) называется
  \emph{предикатом}. Говорят, что предикат \(\varphi\) \emph{выделяет}
  или \emph{покрывает} объект \(x \in X\), если \(\varphi(x) = 1\).
\end{definition}

\begin{definition}
  Предикат называется \emph{закономерностью}, если он выделяет
  достаточно много объектов своего класса и практически не выделяет
  объекты чужого класса.
\end{definition}

Понятие \emph{закономерности} с одной стороны интуитивно, с другой
стороны слишком неточно с математической точки зрения. Формализуем его
как в \cite{voron10logicalgs}.

\begin{itemize}
\item Пусть \(P_c\) --- число объектов класса \(c\) в выборке \(X^l\).
\item Пусть \(p_c(\varphi)\) --- число объектов \(x\) класса \(c\), на
  которых \(\varphi(x) = 1\)
\item Пусть \(N_c\) --- число объектов всех остальных классов \(Y
  \setminus \left\{c\right\}\)
\item Пусть \(n_c(\varphi)\) --- число объектов \(x\) из объектов не
  класса \(c\), на которых \(\varphi(x) = 1\)
\item Обозначим долю негативных объектов среди всех объектов,
  выделяемых \emph{закономерностью}, через \(E_c(\varphi, X^l) =
  \frac{n_c(\varphi)}{p_c(\varphi) + n_c(\varphi)}\)
\item Обозначим долю позитивных объектов, выделяемых
  \emph{закономерностью}, через \(D_c(\varphi, X^l) =
  \frac{p_c(\varphi)}{l}\)
\end{itemize}

\begin{definition}
  Предикат \(\varphi(x)\) называется \emph{логической
    \(\varepsilon\),\(\delta\)-закономерностью} для класса \(c \in
  Y\), если \(E_c(\varphi, X^l) \leq \varepsilon \) и \(D_c(\varphi,
  X^l)\geq\delta\) при заданных достаточно малом \(\varepsilon\) и
  достаточно большом \(\delta\) из отрезка \([0, 1]\)
\end{definition}

Вообще говоря, предикаты \(\varphi(x)\colon X \rightarrow Y\) могут
иметь довольно сложный вид. Например, если в качестве объектов из
\(X\) выступают точки на числовой прямой, то для классификации на
рациональные и иррациональные числа можно взять аналог функции Дирихле
в качестве предиката:

\[
\varphi(x) =
\begin{dcases*}
1 & \(x \in \mathbf{Q}\) \\
0 & \(x \in \mathbf{R} \setminus \mathbf{Q}\)
\end{dcases*}
\]

Для того, чтобы предикаты были более простыми, введем
ограничение на их вид так, как было сделано В.~В.~Рязановым в
\cite{ryazanov07logic}.

\begin{definition}
Пусть \(c_1, c_2 \in R\), \(f_j(x)\) --- j-ый признак объекта \(x\),
тогда \emph{параметрическим элементарным предикатом} будем
называть
\[
P^{c_1, c_2}(f_j(x)) =
\begin{dcases*}
1 & при \(c_1 \leq f_j(x) \leq c_2\) \\
0 & иначе
\end{dcases*}
\]
\end{definition}

Такой предикат называется элементарным, потому что в нем используется
лишь один признак из всего признакового описания объекта.
Параметрическим же он называется потому, что параметры \(c_1, c_2 \in
R\) для каждого отдельного признака \(f_j(x)\) выбираются, вообще
говоря, свои.

Будем считать, что
\emph{логическая \(\varepsilon\),\(\delta\)-закономерность} \(\varphi(x)\)
имеет вид конъюнкции параметрических элементарных предикатов:
\[
\varphi(x) = \land_{f_j(x)} P^{c_1^j, c_2^j}(f_j(x))
\]

Здесь индексы j над параметрами \(c_1^j, c_2^j\) означают, что для
каждого признака могут быть выбраны свои границы. Для логических
закономерностей такого вида есть простая геометрическая интерпретация
--- это построенные по объектам обучающей выборки \(X\) прямоугольные
гиперпараллелепипеды.


\subsection{Обзор литературы}

Лучше, чтобы название этого подраздела было содержательным, например,
общепринятым названием задачи, проблемы или метода, рассматриваемого
в~данной работе.

Перечисляются подходы, методы, факты, на которые существенно опирается
данная работа, но~которые могут быть не~известны широкому кругу
читателей.  Здесь ссылки на литературу обязательны.  Теоремы только
формулируются, но не~доказываются.

Данный раздел преследует две цели.  Во-первых, сделать работу
самодостаточной~--- дать необходимый минимум информации тем читателям,
которые не~очень хорошо ориентируются в~теме, но желают поближе
познакомиться именно с~данной работой.  Во-вторых, облегчить
сопоставление полученных автором результатов с~ранее известными.

\section{Новые подходы и~результаты}

Название этого раздела обязательно надо заменить на~содержательное.
В~этом разделе, как правило, много подразделов.

В~дипломной работе не~стоит делать более двух уровней, достаточно
разделов и~подразделов.  Будете писать диссертацию или монографию~---
сделаете три уровня.

\section{Вычислительные эксперименты}

Цель данного раздела: продемонстрировать, что предложенная теория
работает на практике; показать границы её применимости; рассказать
о~новых экспериментальных фактах.

Чисто теоретические работы могут вообще не~содержать раздела
экспериментов (не~работает, ну и~не~надо~--- зато теория красивая).
Кстати, теоретики имеют право не~догадываться, где, кому и~когда их
теории пригодятся.

\subsection{Исходные данные и~условия эксперимента}
Описывается прикладная задача, параметры анализируемых данных
(например, сколько объектов, сколько признаков, каких они типов),
параметры эксперимента (например, как производился скользящий
контроль).

\subsection{Результаты эксперимента}
Результаты экспериментов представляются в~виде таблиц и~графиков.
Объясняется точный смысл всех обозначений на графиках, строк
и~столбцов в~таблицах.

\subsection{Обсуждение и~выводы}
Приводятся выводы: в~какой степени результаты экспериментов
согласуются с~теорией?  Достигнут ли желаемый результат?  Обнаружены
ли какие-либо факты, не~нашедшие объяснения, и~которые нельзя списать
на «грязный» эксперимент?

Обсуждаются основные отличия предложенных методов от известных ранее.
В~чем их преимущества?  Каковы границы их применимости?  Какие
проблемы удалось решить, а~какие остались открытыми?  Какие возникли
новые постановки задач?

\section{Заключение}

В~квалификационных работах последний раздел нужен для того, чтобы
конспективно перечислить основные результаты, полученные лично
автором.

Результатами, в~частности, являются:
\begin{itemize}
\item
    Предложен новый подход к\dots
\item
    Разработан новый метод\dots, позволяющий\dots
\item
    Доказан ряд теорем, подтверждающих (опровергающих), что\dots
\item
    Проведены вычислительные эксперименты\dots, которые подтвердили /
    опровергли / привели к~новым постановкам задач.
\end{itemize}

Цель данного раздела: доказать квалификацию автора.  Даже беглого
взгляда на заключение должно быть достаточно, чтобы стало ясно: автору
удалось решить актуальную, трудную, ранее не~решённую задачу,
предложенные автором решения обоснованы и~проверены.

Иногда в~Заключении приводится список направлений дальнейших
исследований.

\newpage
Список литературы необходим в~любой научной публикации. В дипломной
работе он~обязателен. Дурным тоном считается: ссылаться на работы
только одного-двух авторов (например, себя или шефа); ссылаться на
слишком малое число работ; ссылаться только на очень старые работы;
ссылаться на работы, которых автор ни разу не видел; ссылаться
на~работы, которые не~упоминаются в~тексте или которые не~имеют
отношения к~данному тексту.

\appendix
\section{Система <<Распознавание>>}
Система <<Распознавание>> по словам разработчиков \cite{recognition06}
является <<универсальной программной системой интеллектуального
анализа данных и прогноза>>. Этот программный продукт работает под
управлением операционной системы семейства Microsoft Windows XP и
выше.

В этом приложении будет показано, как систему <<Распознавание>> можно
использовать на UNIX-подобных операционных системах. Для этого нужно:

\begin{itemize}
  \item Установить свободное программное обеспечение
    \href{https://www.winehq.org/}{Wine}.
  \item Настроить 32-битный префикс командой
    \mint{shell}|WINEARCH=win32 WINEPREFIX=~/.wine32 winecfg|
  \item Установить в созданный префикс недостающую библиотеку
    \mint{shell}|WINEPREFIX=~/.wine32 winetricks mfc42|
  \item Запустить систему <<Распознавание>> командой
    \begin{minted}{shell}
      LC_CTYPE=ru_RU.utf8 WINEARCH=win32 WINEPREFIX=~/.wine32 \
      wine Recognition.exe
    \end{minted}
\end{itemize}

\section{Формат файлов *.tab}

Система <<Распознавание>> использует собственный формат файлов
описания выборки. Здесь приведено описание этого формата:

Первая строка --- заголовок. Представляет из себя числа, разделенные
пробелом. Первое число --- количество признаков каждого объекта
выборки. Второе --- количество классов. Далее идет число \(0\), за
которым идет число объектов первого класса, затем сумма числа объектов
первых двух классов, затем сумма числа объектов первых трех классов и
так далее. Завершает заголовок число, обозначающее пропуск измерения в
данных.

Каждая последующая строка содержит признаковое описание отдельного
объекта, разные классы разделены пустой строкой.

\nocite{
  hastie09elements,
  bishop06pattern,
  recognition06,
  zhuravlev78prob33,
  shlezinger04ten,
  boucheron05theory
}
\printbibliography

\end{document}
